\documentclass[english]{article}
\usepackage[T1]{fontenc}
\usepackage[utf8]{inputenc}
\usepackage{amsthm}
\usepackage{amsmath}
\usepackage{amsfonts}
\usepackage{listings}
\usepackage{babel}

\newtheorem{thm}{\protect\theoremname}
\theoremstyle{definition}
\newtheorem*{example*}{\protect\examplename}
\theoremstyle{definition}
\newtheorem{example}[thm]{\protect\examplename}
\providecommand{\examplename}{Ejemplo}
\newtheorem*{defn*}{\protect\definitionname}
\providecommand{\definitionname}{Definición}


\begin{document}

\title{Control de concurrencia y recuperación en bases de datos}


\section{Transacciones y serializabilidad}

Construiremos un modelo para estudiar los problemas de concurrencia en bases
de datos. En este modelo, veremos a la base de datos como un conjunto de
ítems. Un ítem puede ser un atributo, una tupla o una relación entera; los
denominaremos con letras: $A$, $B$, $X$, $Y$.

\subsection{Transacciones}

\begin{defn*}
Llamaremos \textbf{transacción} a toda ejecución de un programa que accede a
la base de datos. El programa puede ejecutarse varias veces, y cada ejecución
del programa representa una nueva transacción $T_i$.

La transacción es una sucesión de acciones (u operaciones), que representan
pasos atómicos. Estas acciones pueden ser:

\begin{itemize}
    \item Leer un ítem X ($r_i[X]$).
    \item Escribir un ítem X ($w_i[X]$).
    \item \emph{Abort} de la transacción $T_i$ ($a_i$).
    \item \emph{Commit} de la transacción $T_i$ ($c_i$).
\end{itemize}

Luego, una definición más precisa de transacción es la siguiente: $T_i$ es una
transacción si y sólo si:

$$T_i \subseteq \
    \{r_i[X], w_i[X], a_i, c_i: \
        X \mbox{ es un ítem, } i \in \mathbb{N}\}$$
\end{defn*}

El modelo en el que encaja la definición anterior es el llamado \emph{modelo
read/write} o \emph{modelo sin locking}. En él, las $T_i$ se ejecutan en forma
concurrente (entrelazada) y esto, veremos luego, puede llevar a la ocurrencia
de interferencias. Asimismo, pueden ocurrir fallas en medio de la ejecución de
la transacción. Trataremos esto último más adelante, cuando hablemos de
recuperación.

Dos problemas clásicos que pueden presentarse son conocidos como \emph{lost
update} y \emph{dirty read}.

\subsection{Lost update}

El \emph{lost update} o actualización perdida ocurre cuando la actualización
hecha por una transacción $T_1$ se pierde a causa de la acción de otra
transacción $T_2$ sobre el mismo ítem; ambas transacciones leen el mismo valor
anterior del ítem y luego lo actualizan en forma sucesiva.

\begin{example}
Supongamos el programa $P$:
\begin{enumerate}
    \item $Read(A)$
    \item $A := A + 1$
    \item $Write(A)$
\end{enumerate}

La función $Read(A)$ copia el valor del ítem $A$ en disco a la variable local
$A$. De manera inversa, $Write(A)$ copia el valor de la variable local $A$ al
disco. $A := A + 1$ actualiza el valor de la variable local.

Supongamos ahora dos ejecuciones de $P$, $T_1$ y $T_2$, con el siguiente
entrelazamiento:

\vspace{10pt}

\begin{tabular}{ l l l }
  $T_1$         & $T_2$         & A (en disco)  \\
  \hline
  $Read(A)$     &               & 5             \\
                & $Read(A)$     & 5             \\
  $A := A + 1$  &               & 5             \\
                & $A := A + 1$  & 5             \\
                & $Write(A)$    & 6             \\
  $Write(A)$    &               & 6             \\
\end{tabular}

\vspace{10pt}

Notar que ocurre la siguiente secuencia:
\begin{enumerate}
    \item $T_1$ y $T_2$ leen el valor de $A$ desde el disco y lo almacenan en
        su variable local homónima.
    \item $T_1$ y $T_2$ actualizan el valor de su variable local $A$.
    \item $T_2$ escribe el valor actualizado de $A$ en disco, y luego lo hace
        $T_1$.
\end{enumerate}

Naturalmente, uno esperaría que, luego de la ejecución de ambas transacciones,
el valor de $A$ en disco fuera incrementado dos veces. Sin embargo, al
ejecutar las transacciones concurrentemente, puede ocurrir que la
actualización de una de las dos transacciones se pierda, como se observa en el
ejemplo.
\end{example}


\end{document}
