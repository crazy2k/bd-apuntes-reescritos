%% LyX 2.0.3 created this file.  For more info, see http://www.lyx.org/.
%% Do not edit unless you really know what you are doing.
\documentclass[english]{article}
\usepackage[T1]{fontenc}
\usepackage[utf8]{inputenc}
\usepackage{amsthm}
\usepackage{amsmath}
\usepackage{listings}
\usepackage{babel}

\newtheorem{thm}{\protect\theoremname}
\theoremstyle{definition}
\newtheorem*{example*}{\protect\examplename}
\theoremstyle{definition}
\newtheorem{example}[thm]{\protect\examplename}
\providecommand{\examplename}{Ejemplo}
\newtheorem*{defn*}{\protect\definitionname}
\providecommand{\definitionname}{Definición}



\begin{document}

\title{Normalización}


\section{Modelo Relacional}

Usaremos la siguiente notación:
\begin{itemize}
	\item $R, S$: Esquemas de relación.
	\item $A_1, A_2, \dots, A_n$: Atributos de un esquema de relación (también
		podremos usar $A, B, C$ y demás letras, de las primeras del alfabeto,
		para lo mismo).
	\item $r, s$: Instancias de esquemas de relación.
	\item $t_1, t_2, \dots$: Tuplas de una instancia.
	\item $t(A_1, A_2, \dots, A_n)$: Proyección de los atributos $A_1, A_2,
		\dots, A_n$ para la tupla $t$.
	\item $X$, $Y$: Conjuntos de atributos de un esquema de relación.
\end{itemize}


\section{Dependencias Funcionales}


\subsection{Definición de Dependencia Funcional}

\begin{defn*}
Sea $R$ un esquema de relación. Sean $X$ e $Y$ conjuntos no vaciós de
atributos de $R$ (pueden no ser disjuntos). Decimos que $X$
\emph{determina funcionalmente} a $Y$ (notado $X \rightarrow Y$) en $R$, si
para toda instancia $r$ de $R$, se verifica que:

\begin{center}
si $t_1(X) = t_2(X)$, entonces $t_1(Y) = t_2(Y)$
\end{center}

siendo $t_1$ y $t_2$ tuplas cualesquiera de $r$.
\end{defn*}

Si una instancia $r$ cumple con todas las dependencias funcionales, decimos
que $r$ es \emph{legal}. Las dependencias funcionales son establecidas por el
diseñador de la base de datos.

\begin{example}
\label{ej:df}
Supongamos que queremos registrar, para una Facultad, los datos personales de
los alumnos, las materias en las que se inscribieron y los exámenes que
rindieron. Para esto definimos el siguiente esquema de relación:

$$Facultad(lu, nombre, materia, ifec, efec, nota)$$

y el siguiente conjunto de dependencias funcionales:
\begin{itemize}
	\item $lu \rightarrow nombre$ (no puede haber dos alumnos con el mismo
		$lu$)
	\item $lu, materia \rightarrow ifec$ (el alumno se puede inscribir una
		sola vez en cada materia)
	\item $lu, materia, efec \rightarrow nota$ (hay una sola nota por examen)
\end{itemize}

Nota: Al no estar $lu, materia \rightarrow efec$, se puede rendir varias veces
la misma materia.
\end{example}


\subsection{Inferencias de un conjunto de dependencias funcionales}

Sean $F$ y $f$ conjuntos de dependencias funcionales. Decimos que $F$
\emph{infiere} $f$ (notado $F \models f$) si y sólo si toda instancia $r$ que
satisface $F$ también satisface $f$.

\begin{example}
Llamemos $F$ al conjunto de dependencias funcionales dado en el ejemplo
\ref{ej:df}. Podemos inferir:
$F \models lu, nombre \rightarrow nombre, ifec$
\end{example}


\subsection{Reglas de inferencia (axiomas de Armstrong)}

\begin{enumerate}
	\item \label{regla:refl} Reflexividad: Si $Y \subseteq X$, entonces
		$X \rightarrow Y$.
	\item Aumento: Para cualquier $W$, si $X \rightarrow Y$, entonces
		$XW \rightarrow WY$.
	\item Transitividad: Si $X \rightarrow Y$ e $Y \rightarrow Z$, entonces
		$X \rightarrow Z$.
\end{enumerate}

De la regla \ref{regla:refl}, inferimos que siempre se cumple
$X \rightarrow X$.

Notar que no necesariamente si se cumple $X \rightarrow Y$, debe cumplirse
$Y \rightarrow X$.


\subsection{Clausura de un conjunto de dependencias funcionales}

La \emph{clausura de un conjunto de dependencias funcionales} $F$, notada
$F^+$, es el conjunto de todas las dependencias funcionales que pueden
inferirse a partir de $F$ aplicando los axiomas. Es decir,

$$F^+ = \{X \rightarrow Y: F \models X \rightarrow Y\}$$

\begin{example}
Sea el esquema de relación $R(A, B)$ y el conjunto de dependencias funcionales
$F = \{A \rightarrow B\}$. Calculamos la clausura de $F$:

$$F^+ = \{A \rightarrow B, A \rightarrow A, B \rightarrow B, AB \rightarrow A,
	AB \rightarrow B, AB \rightarrow AB, A \rightarrow AB\}$$
\end{example}

\begin{example}
Sea el esquema de relación $R(A, B, C)$ y el conjunto de dependencias funcionales
$F = \{AB \rightarrow C, C \rightarrow B\}$. Calculamos la clausura de $F$:

\begin{multline*}
F^+ = \{A \rightarrow A, AB \rightarrow A, AC \rightarrow A,
	ABC \rightarrow A, B \rightarrow B, AB \rightarrow B, \\
	BC \rightarrow B, ABC \rightarrow B, ABC \rightarrow B,
	C \rightarrow C\, AC \rightarrow C, BC \rightarrow C, \\
	ABC \rightarrow C, AB \rightarrow AB, ABC \rightarrow AB,
	AC \rightarrow AC, ABC \rightarrow AC, \\
	BC \rightarrow BC, ABC \rightarrow BC, ABC \rightarrow ABC,
	AB \rightarrow C, AB \rightarrow AC, \\
	AB \rightarrow BC, AB \rightarrow ABC, C \rightarrow B,
	C \rightarrow BC, AC \rightarrow B, AC \rightarrow AB \}
\end{multline*}
\end{example}

\subsection{Reglas adicionales}
\begin{enumerate}
	\setcounter{enumi}{3}
	\item Unión: Si $X \rightarrow Y$ y $X \rightarrow Z$, entonces
		$X \rightarrow YZ$.
	\item Pseudotransitividad: Para cualquier $W$, si $X \rightarrow Y$ e
		$YW \rightarrow Z$, entonces $XW \rightarrow Z$.
	\item Descomposición: Si $X \rightarrow YZ$, entonces $X \rightarrow Y$ y
	$X \rightarrow Z$.
\end{enumerate}

Estas reglas adicionales se demuestran aplicando los axiomas.

\begin{example}
\end{example}

\begin{example}
\end{example}

\subsection{Clausura de un conjunto de atributos}
La clausura del conjunto de atributos $X$, de un esquema de relación $R$, con
respecto a un conjunto de dependencias funcionales $F$, se nota $X+$ y se
define:

$$X^+ = \{A \in R: F \models X \rightarrow A\}$$

Una forma de calcular $X^+$ es [...]

\subsection{Propiedades de la clausura de un conjunto de atributos}

Dados [...]

\newpage

\section{Apunte para el parcial}

\begin{defn*}
	Un \textbf{atributo primo} es aquel que es miembro de una clave candidata.
\end{defn*}

\begin{defn*}
	Decimos que $X \rightarrow Y$ es una \textbf{dependencia funcional
	parcial} (o que $Y$ depende parcialmente de $X$) si para algún
	$Z \subset X$, se verifica que $Z \rightarrow Y$.

	Si no, decimos que es una \textbf{dependencia funcional total} (o que $Y$
	depende totalmente de $X$).
\end{defn*}

\subsection{Formas normales}

\begin{itemize}
	\item \textbf{1FN:} Se cumple siempre en los casos que vemos. Sólo exige
		que los campos tengan valores atómicos, es decir, no sean listas o
		conjuntos.
	\item \textbf{2FN:} Si todo atributo no primo (no miembro de alguna clave)
		es totalmente dependiente de todas las claves de la relación.
		
		O sea, tengo que tomar una a una las claves $X$ de la relación y
		verificar que para cada atributo no primo $Y$, se cumpla
		$X \rightarrow Y$ y además $X$ no tenga atributos redundantes.
	\item \textbf{3FN:} Si para cada DF no trivial $X \rightarrow A$:
		\begin{itemize}
			\item $X$ es superclave, o
			\item $A$ es un atributo primo (está en alguna clave)
		\end{itemize}
	\item \textbf{FNBC:} Lo mismo que \textbf{3FN} pero sin el último ítem.
\end{itemize}

\subsection{Cubrimiento minimal}

\begin{enumerate}
	\item \textbf{Usar descomposición para tener las DFs con un
		solo atributo a derecha.} \\
		Por ejemplo, si tengo la DF $X \rightarrow YZ$, la reemplazo por
		$X \rightarrow Y$ y $X \rightarrow Z$.
	\item \textbf{Minimizar lados izquierdos viendo qué atributos se pueden
		sacar preservando equivalencia con $F^+$.} Para esto, tomo DFs con
		varios atributos en su lado izquierdo y voy sacando atributos de a uno
		y haciendo la clausura sin ese atributo. La clausura la hago usando
		todas las dependencias funcionales originales, incluyendo la
		dependencia que estoy analizando (completa). Si lo que estaba en el
		lado derecho aparece en la clausura, entonces ese atributo era
		redundante. \\
		Por ejemplo, si tengo $CE \rightarrow A$ y obtengo que $C^+ = ABC$,
		entonces reemplazo $CE \rightarrow A$ por $C \rightarrow A$ (el
		atributo $E$ era redundante).
	\item \textbf{Quitar DFs redundantes.} Para cada DF $X \rightarrow Y$,
		calculamos $X^+_{\bullet - (X \rightarrow Y)}$, esto es, la clausura
		de $X$ usando todas las DFs excepto $X \rightarrow Y$. Si en el
		resultado se encuentra $Y$, entonces la DF $X \rightarrow Y$ era
		redundante y podemos eliminarla.
\end{enumerate}


\subsection{Descomposición en 3FN SPI y SPDF}

Parto de un cubrimiento minimal.

\begin{enumerate}
	\item Convertir cada DF $X \rightarrow A$ en un esquema $XA$ (mismo
		lado izquierdo, se juntan)
	\item Si ningún esquema contiene una clave, agregar uno con los
		atributos de alguna clave.
	\item Eliminar esquemas redundantes (los contenidos totalmente en otro)
\end{enumerate}


\subsection{Descomposición en FNBC SPI (quizás no SPDF)}

\begin{enumerate}
	\item Computar claves de $R$.
	\item Repetir hasta que todas las relaciones estén en FNBC:
		\begin{enumerate}
			\item Tomar algún $R'$ con $A \rightarrow B$ que viole FNBC.
			\item Descomponer $R'$ en $R_1(A, B)$ y $R_2 = R - B$.
			\item Computar DFs de $R_1$ y $R_2$. Para esto, se debe aplicar el
                siguiente algoritmo (reemplazar $i$ por 1 o 2 según para cuál
                se esté proyectando):
                \begin{enumerate}
                    \item $T = \emptyset$
                    \item Para cada $X$, subconjunto de $R_i$:
                        \begin{enumerate}
                            \item Computar $X^+$ usando las DFs de $R$.
                            \item Agregar a $T$ todas las DFs no triviales
                                $X \rightarrow A$ tales que $A$ está en $X^+$
                                y es un atributo de $R_i$.
                        \end{enumerate}
                \end{enumerate}
			\item Computar claves de $R_1$ y $R_2$.
		\end{enumerate}
\end{enumerate}


\subsection{Verificación de SPI}

Usar algoritmo del Tableau.

\subsection{Verificación de SPDF}

Para cada DF $X \rightarrow Y$, analizo si se preserva:
\begin{enumerate}
	\item Si $XY \in R_i$, entonces se preserva.
	\item Si no:
		\begin{enumerate}
			\item $Z = X$
			\item Mientras $Z$ cambie:\\
				Para $i = 1, \dots, k$:
				\begin{enumerate}
					\item $Z = Z \cup ((Z \cap R_i)^+ \cap R_i)$
					\item Si $Y \in Z$, entonces se preserva.
				\end{enumerate}
			\item No se preserva.
		\end{enumerate}
\end{enumerate}


\part{Ejercicios de parciales}

\section{Primer Parcial: 1C2013}

\subsection{Ejercicio 2}

\subsubsection{Inciso a)}

$$\rho(pa1, Participo \bowtie Actor)$$
$$\rho(pa2, Participo \bowtie Actor)$$
$$\rho(pa3, Participo \bowtie Actor)$$
$$\rho(pa1ypa2, pa1
	\bowtie_{pa1.idPeli = pa2.idPeli AND pa1.idPais \not= pa2.idPais} pa2)$$
$$\rho(pa1ypa2ypa3, pa1ypa2
	\bowtie_{pa1ypa2.idPeli = pa3.idPeli AND pa1.idPais \not= pa3.idPais AND
		pa2.idPais \not= pa3.idPais} pa3)$$
$$\rho(peliConAct2Nac, \Pi_{idPeli}(pa1ypa2) - \Pi_{idPeli}(pa1ypa2ypa3))$$
$$\rho(peli1, peliConAct2Nac \bowtie Pelicula)$$
$$\rho(peli2, peliConAct2Nac \bowtie Pelicula)$$
$$\rho(peliNoMayor, \Pi_{peli1.idPeli}
	(peli1 \bowtie_{peli1.duracion < peli2.duracion} peli2))$$
$$\Pi_{nombre}((peliConAct2Nac - \Pi_{idPeli}peliNoMayor) \bowtie Pelicula)$$

\subsubsection{Inciso b)}

\begin{lstlisting}
SELECT a.nombre
FROM Actor a
WHERE
    100 < (SELECT COUNT(*)
        FROM Actor a2
        WHERE a2.idPais = a.idPais)
    AND
    NOT EXISTS (
        SELECT 1
        FROM Genero g
        WHERE
            NOT EXISTS (
                SELECT 1
                FROM Pelicula p, Participo par
                WHERE p.idPeli = par.idPeli AND
                    par.idActor = a.idActor AND
                    p.idGenero = g.idGenero
            )
    )
\end{lstlisting}

\subsection{Ejercicio 3}

\subsubsection{Inciso 1}

$(AF)^+ = ABDCEFGH$, por lo que $AF$ es clave. Y es la única. Esto es porque
$A$ y $F$ no están en ningún lado derecho y, por lo tanto, son parte de todas
las claves posibles. Como cualquier otra superclave tendrá más atributos, $AF$
es la única clave.

\subsubsection{Inciso 2}

\begin{itemize}
    \item \textbf{¿FNBC?} No. Por ejemplo, en $H \rightarrow G$, $H$ no es
        superclave.
    \item \textbf{¿3FN?} No. Por ejemplo, en $H \rightarrow G$, $H$ no es
        superclave y $G$ no es un atributo primo (no está en ninguna clave).
    \item \textbf{¿2FN?} No. Tomemos el atributo no primo $E$. Tenemos que
        $AF \rightarrow E$, porque $AF$ es clave. Sin embargo, tenemos que
        $F \rightarrow E$. Luego, $E$ depende parcialmente de la clave $AF$.
    \item \textbf{¿1FN?} Sí.
\end{itemize}

\subsubsection{Inciso 3}

\begin{enumerate}
    \item
        \begin{itemize}
            \item $AB \rightarrow H$
            \item $AB \rightarrow G$
            \item $B \rightarrow D$
            \item $BD \rightarrow C$
            \item $E \rightarrow G$
            \item $E \rightarrow C$
            \item $F \rightarrow D$
            \item $F \rightarrow B$
            \item $F \rightarrow E$
            \item $H \rightarrow G$
        \end{itemize}
    \item
        \begin{itemize}
            \item Caso $AB \rightarrow H$. $A^+ = A$. $B^+ = BCD$. No consigo
                $H$.
            \item Caso $AB \rightarrow G$. $A^+ = A$. $B^+ = BCD$. No consigo
                $G$.
            \item Caso $BD \rightarrow C$. $B^+ = BCD$. Consigo $C$. Reemplazo
                la DF por $B \rightarrow C$.
        \end{itemize}
    \item
        \begin{itemize}
            \item $(AB)^+_{\bullet - (AB \rightarrow H)} = ABCDG$. No consigo
                $H$.
            \item $(AB)^+_{\bullet - (AB \rightarrow G)} = ABCDGH$. Consigo
                $G$. Esta DF es redundante.
            \item $B^+_{\bullet - (B \rightarrow D)} = BC$. No consigo $D$.
            \item $B^+_{\bullet - (B \rightarrow C)} = BD$. No consigo $C$.
            \item $E^+_{\bullet - (E \rightarrow G)} = EC$. No consigo $G$.
            \item $E^+_{\bullet - (E \rightarrow C)} = EG$. No consigo $C$.
            \item $F^+_{\bullet - (F \rightarrow D)} = BCDEFG$. Consigo $D$.
                Esta DF es redundante.
            \item $F^+_{\bullet - (F \rightarrow B)} = CDEFG$. No consigo $B$.
            \item $F^+_{\bullet - (F \rightarrow E)} = BCDF$. No consigo $E$.
            \item $H^+_{\bullet - (H \rightarrow G)} = H$. No consigo $G$.
        \end{itemize}
\end{enumerate}

La cobertura minimal queda:
\begin{itemize}
    \item $AB \rightarrow H$
    \item $B \rightarrow D$
    \item $B \rightarrow C$
    \item $E \rightarrow G$
    \item $E \rightarrow C$
    \item $F \rightarrow B$
    \item $F \rightarrow E$
    \item $H \rightarrow G$
\end{itemize}

\subsubsection{Inciso 4}

\begin{enumerate}
    \item $\{ABH, BDC, EGC, FBE, HG\}$
    \item $\{ABH, BDC, EGC, FBE, HG, AF\}$
    \item $\{ABH, BDC, EGC, FBE, HG, AF\}$
\end{enumerate}

\subsubsection{Inciso 5}

Las relaciones de 2 atributos, están en FNBC. Para las otras, tengo que
proyectar las DFs iniciales y verificarlo.
\begin{itemize}
    \item Caso $ABH$. Calculo:
        \begin{itemize}
            \item $A^+=A$
            \item $B^+=BCD$
            \item $H^+=GH$
            \item $(AB)^+=ABCDGH$
            \item $(AH)^+=AHG$
            \item $(BH)^+=BCDGH$
        \end{itemize}
        Las dependencias serán:
        \begin{itemize}
            \item $AB \rightarrow H$
        \end{itemize}
        Como $AB$ es superclave, $ABH$ está en FNBC.
    \item Caso $BCD$. Calculo:
        \begin{itemize}
            \item $B^+=BCD$
            \item $C^+=C$
            \item $D^+=D$
            \item $(BC)^+=BCD$
            \item $(BD)^+=BCD$
            \item $(CD)^+=CD$
        \end{itemize}
        Las dependencias serán:
        \begin{itemize}
            \item $B \rightarrow C$
            \item $B \rightarrow D$
            \item $BC \rightarrow D$
            \item $BD \rightarrow C$
        \end{itemize}
        Como todos los lados izquierdos son superclave, $BCD$ está en FNBC.
    \item TODO: Faltaría hacer lo mismo con $EGC$ y $FBE$.
\end{itemize}

\end{document}
